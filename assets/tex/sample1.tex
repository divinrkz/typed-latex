\documentclass{article}
\usepackage{amsmath}

\begin{document}

\textbf{Theorem}: The sum of two even numbers is even.

\textbf{Proof}: Let $a$ and $b$ be two even numbers. By definition of even numbers, there exist integers $k_1$ and $k_2$ such that:

$a = 2k_1 \quad \text{and} \quad b = 2k_2$

Now, consider the sum of $a$ and $b$:

    \textbf{Hello World!} Today I am learning \LaTeX. %notice how the command will end at the first non-alphabet charecter such as the . after \LaTeX
     \LaTeX{} is a great program for writing math. I can write in line math such as. Let $x \in \mathbb{Z}$ and  $x \in \mathbb{Z}$ be arbitrary %$ tells LaTexX to compile as math
     . I can also give equations their own space: 
    \begin{equation}  \label{eq1} % Creates an equation environment and is compiled as math
        \begin{split}            
            \gamma^2+\theta^2=\omega^2 \\
            \gamma = \sqrt{\omega^2 - \theta^2}
        \end{split}
    \end{equation}
    If I do not leave any blank lines \LaTeX{} will continue  this text without making it into a new paragraph.  Notice how there was no indentation in the text after equation (1).  
    Also notice how even though I
    $a \geq b $  hit enter after that sentence and here 
     \LaTeX{} formats the sentence without any break.  Also   look  how      it   doesn't     matter          how    many  spaces     I put     between       my    words.
    
    For a new paragraph I can leave a blank space in my code.  $${x \vert x} \in L$$

Since $k_1 + k_2$ is an integer, it follows that $a + b$ is divisible by 2. Therefore, $a + b$ is even.

\end{document}

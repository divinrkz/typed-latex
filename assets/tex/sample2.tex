
% This is a simple sample document.  For more complicated documents take a look in the exercise tab. Note that everything that comes after a % symbol is treated as comment and ignored when the code is compiled.

\documentclass{article} % \documentclass{} is the first command in any LaTeX code.  It is used to define what kind of document you are creating such as an article or a book, and begins the document preamble

\usepackage{amsmath} % \usepackage is a command that allows you to add functionality to your LaTeX code

\title{Sample 2} % Sets article title
\author{My Name} % Sets authors name
\date{\today} % Sets date for date compiled
% The preamble ends with the command \begin{document}
\begin{document}

\title{A Simple Proof in Number Theory}
\author{Author Name}
\date{\today}
\maketitle

\section*{Proof}

\textbf{Theorem:} For any integer $a$, if $a \leq b$, then $a + c \leq b + c$ for any integer $c$.

\textbf{Proof:}

Let $a, b,$ and $c$ be arbitrary integers such that $a \leq b$. We need to show that $a + c \leq b + c$.

Given $a \leq b$, we can add the integer $c$ to both sides of the inequality:
\begin{equation}
    \begin{split}
        a + c \leq b + c
        a \leq b \geq c < d
    \end{split}
\end{equation}

This follows from the basic properties of inequalities, where adding the same number to both sides of an inequality preserves the inequality.

Thus, we have shown that if $a \leq b$, then $a + c \leq b + c$ for any integer $c$. \n

This just a line to add all the inequalities Let $x = 3$ and $y = 4$. So, $x \neq y$. Also, we could say $y > x$ and $y \geq x$ is also valid.

Also, $n = m = p$ and $0 < x < 16$ and for $ y \in \mathbb{N}$

$a \leq b \geq c < d$

$$a + b \\ b + c$$

\end{document}
